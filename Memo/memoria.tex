\documentclass[a4paper,12pt,twoside]{memoir}

% Castellano
\usepackage[spanish,es-tabla]{babel}
\selectlanguage{spanish}
\usepackage[utf8]{inputenc}
\usepackage[T1]{fontenc}
\usepackage{lmodern} % Scalable font
\usepackage{microtype}
\usepackage{placeins}
\usepackage{float}
\usepackage{caption}

\RequirePackage{booktabs}
\RequirePackage[table]{xcolor}
\RequirePackage{xtab}
\RequirePackage{multirow}
\usepackage{graphicx}
\usepackage{longtable}
\usepackage{adjustbox}

% Links
\PassOptionsToPackage{hyphens}{url}\usepackage[colorlinks]{hyperref}
\hypersetup{
	allcolors = {red}
}

% Ecuaciones
\usepackage{amsmath}

% Rutas de fichero / paquete
\newcommand{\ruta}[1]{{\sffamily #1}}

% Párrafos
\nonzeroparskip

% Huérfanas y viudas
\widowpenalty100000
\clubpenalty100000

% Imágenes

% Comando para insertar una imagen en un lugar concreto.
% Los parámetros son:
% 1 --> Ruta absoluta/relativa de la figura
% 2 --> Texto a pie de figura
% 3 --> Tamaño en tanto por uno relativo al ancho de página
\usepackage{graphicx}
\newcommand{\imagen}[3]{
	\begin{figure}[!h]
		\centering
		\includegraphics[width=#3\textwidth]{#1}
		\caption{#2}\label{fig:#1}
	\end{figure}
	\FloatBarrier
}

% Comando para insertar una imagen sin posición.
% Los parámetros son:
% 1 --> Ruta absoluta/relativa de la figura
% 2 --> Texto a pie de figura
% 3 --> Tamaño en tanto por uno relativo al ancho de página
\newcommand{\imagenflotante}[3]{
	\begin{figure}
		\centering
		\includegraphics[width=#3\textwidth]{#1}
		\caption{#2}\label{fig:#1}
	\end{figure}
}

% El comando \figura nos permite insertar figuras comodamente, y utilizando
% siempre el mismo formato. Los parametros son:
% 1 --> Porcentaje del ancho de página que ocupará la figura (de 0 a 1)
% 2 --> Fichero de la imagen
% 3 --> Texto a pie de imagen
% 4 --> Etiqueta (label) para referencias
% 5 --> Opciones que queramos pasarle al \includegraphics
% 6 --> Opciones de posicionamiento a pasarle a \begin{figure}
\newcommand{\figuraConPosicion}[6]{%
  \setlength{\anchoFloat}{#1\textwidth}%
  \addtolength{\anchoFloat}{-4\fboxsep}%
  \setlength{\anchoFigura}{\anchoFloat}%
  \begin{figure}[#6]
    \begin{center}%
      \Ovalbox{%
        \begin{minipage}{\anchoFloat}%
          \begin{center}%
            \includegraphics[width=\anchoFigura,#5]{#2}%
            \caption{#3}%
            \label{#4}%
          \end{center}%
        \end{minipage}
      }%
    \end{center}%
  \end{figure}%
}

%
% Comando para incluir imágenes en formato apaisado (sin marco).
\newcommand{\figuraApaisadaSinMarco}[5]{%
  \begin{figure}%
    \begin{center}%
    \includegraphics[angle=90,height=#1\textheight,#5]{#2}%
    \caption{#3}%
    \label{#4}%
    \end{center}%
  \end{figure}%
}
% Para las tablas
\newcommand{\otoprule}{\midrule [\heavyrulewidth]}
%
% Nuevo comando para tablas pequeñas (menos de una página).
\newcommand{\tablaSmall}[5]{%
 \begin{table}[H]
  \begin{center}
   \rowcolors {2}{gray!35}{}
   \begin{tabular}{#2}
    \toprule
    #4
    \otoprule
    #5
    \bottomrule
   \end{tabular}
   \caption{#1}
   \label{tabla:#3}
  \end{center}
 \end{table}
}

%
% Nuevo comando para tablas pequeñas (menos de una página).
\newcommand{\tablaSmallSinColores}[5]{%
 \begin{table}
  \begin{center}
   \begin{tabular}{#2}
    \toprule
    #4
    \otoprule
    #5
    \bottomrule
   \end{tabular}
   \caption{#1}
   \label{tabla:#3}
  \end{center}
 \end{table}
}

\newcommand{\tablaApaisadaSmall}[5]{%
\begin{landscape}
  \begin{table}
   \begin{center}
    \rowcolors {2}{gray!35}{}
    \begin{tabular}{#2}
     \toprule
     #4
     \otoprule
     #5
     \bottomrule
    \end{tabular}
    \caption{#1}
    \label{tabla:#3}
   \end{center}
  \end{table}
\end{landscape}
}

%
% Nuevo comando para tablas grandes con cabecera y filas alternas coloreadas en gris.
\newcommand{\tabla}[6]{%
  \begin{center}
    \tablefirsthead{
      \toprule
      #5
      \otoprule
    }
    \tablehead{
      \multicolumn{#3}{l}{\small continúa desde la página anterior}\\
      \toprule
      #5
      \otoprule
    }
    \tabletail{
      \hline
      \multicolumn{#3}{r}{\small continúa en la página siguiente}\\
    }
    \tablelasttail{
      \hline
    }
    \bottomcaption{#1}
    \rowcolors {2}{gray!35}{}
        \begin{tabular}{#2}
          #6
          \bottomrule
        \end{tabular}% 
    \label{tabla:#4}
  \end{center}

}


% Nuevo comando para tablas grandes con cabecera.
\newcommand{\tablaSinColores}[6]{%
  \begin{center}
    \tablefirsthead{
      \toprule
      #5
      \otoprule
    }
    \tablehead{
      \multicolumn{#3}{l}{\small\sl continúa desde la página anterior}\\
      \toprule
      #5
      \otoprule
    }
    \tabletail{
      \hline
      \multicolumn{#3}{r}{\small\sl continúa en la página siguiente}\\
    }
    \tablelasttail{
      \hline
    }
    \bottomcaption{#1}
    \begin{xtabular}{#2}
      #6
      \bottomrule
    \end{xtabular}
    \label{tabla:#4}
  \end{center}
}

%
% Nuevo comando para tablas grandes sin cabecera.
\newcommand{\tablaSinCabecera}[5]{%
  \begin{center}
    \tablefirsthead{
      \toprule
    }
    \tablehead{
      \multicolumn{#3}{l}{\small\sl continúa desde la página anterior}\\
      \hline
    }
    \tabletail{
      \hline
      \multicolumn{#3}{r}{\small\sl continúa en la página siguiente}\\
    }
    \tablelasttail{
      \hline
    }
    \bottomcaption{#1}
  \begin{xtabular}{#2}
    #5
   \bottomrule
  \end{xtabular}
  \label{tabla:#4}
  \end{center}
}



\definecolor{cgoLight}{HTML}{EEEEEE}
\definecolor{cgoExtralight}{HTML}{FFFFFF}

%
% Nuevo comando para tablas grandes sin cabecera.
\newcommand{\tablaSinCabeceraConBandas}[5]{%
  \begin{center}
    \tablefirsthead{
      \toprule
    }
    \tablehead{
      \multicolumn{#3}{l}{\small\sl continúa desde la página anterior}\\
      \hline
    }
    \tabletail{
      \hline
      \multicolumn{#3}{r}{\small\sl continúa en la página siguiente}\\
    }
    \tablelasttail{
      \hline
    }
    \bottomcaption{#1}
    \rowcolors[]{1}{cgoExtralight}{cgoLight}

  \begin{xtabular}{#2}
    #5
   \bottomrule
  \end{xtabular}
  \label{tabla:#4}
  \end{center}
}



\graphicspath{ {./img/} }

% Capítulos
\chapterstyle{bianchi}
\newcommand{\capitulo}[2]{
	\setcounter{chapter}{#1}
	\setcounter{section}{0}
	\setcounter{figure}{0}
	\setcounter{table}{0}
	\chapter*{\thechapter.\enskip #2}
	\addcontentsline{toc}{chapter}{\thechapter.\enskip #2}
	\markboth{#2}{#2}
}

% Apéndices
\renewcommand{\appendixname}{Apéndice}
\renewcommand*\cftappendixname{\appendixname}

\newcommand{\apendice}[1]{
	%\renewcommand{\thechapter}{A}
	\chapter{#1}
}

\renewcommand*\cftappendixname{\appendixname\ }

% Formato de portada
\makeatletter
\usepackage{xcolor}
\newcommand{\tutor}[1]{\def\@tutor{#1}}
\newcommand{\course}[1]{\def\@course{#1}}
\definecolor{cpardoBox}{HTML}{E6E6FF}
\def\maketitle{
  \null
  \thispagestyle{empty}
  % Cabecera ----------------
\noindent\includegraphics[width=\textwidth]{cabecera}\vspace{1cm}%
  \vfill
  % Título proyecto y escudo informática ----------------
  \colorbox{cpardoBox}{%
    \begin{minipage}{.8\textwidth}
      \vspace{.5cm}\Large
      \begin{center}
      \textbf{TFG del Grado en Ingeniería Informática}\vspace{.6cm}\\
      \textbf{\LARGE\@title{}}
      \end{center}
      \vspace{.2cm}
    \end{minipage}

  }%
  \hfill\begin{minipage}{.20\textwidth}
    \includegraphics[width=\textwidth]{escudoInfor}
  \end{minipage}
  \vfill
  % Datos de alumno, curso y tutores ------------------
  \begin{center}%
  {%
    \noindent\LARGE
    Presentado por \@author{}\\ 
    en Universidad de Burgos --- \@date{}\\
    Tutor: \@tutor{Carlos López Nozal}\\
    Tutor: \@tutor{Raúl Marticorena Sánchez}\\

  }%
  \end{center}%
  \null
  \cleardoublepage
  }
\makeatother

\newcommand{\nombre}{Bilal Azar El Mourabit} %%% cambio de comando

% Datos de portada
\title{Evaluación de calidad de cursos implementados en la versión de Moodle 4.x}
\author{\nombre}
\tutor{}
\date{\today}

\begin{document}

\maketitle


\newpage\null\thispagestyle{empty}\newpage


%%%%%%%%%%%%%%%%%%%%%%%%%%%%%%%%%%%%%%%%%%%%%%%%%%%%%%%%%%%%%%%%%%%%%%%%%%%%%%%%%%%%%%%%
\thispagestyle{empty}


\noindent\includegraphics[width=\textwidth]{cabecera}\vspace{1cm}

\noindent D. Carlos López Nozal, profesor del departamento de Ingeniería
Informática, área de Lenguajes y Sistemas Informáticos.

\noindent D. Raúl Marticorena Sánchez, profesor del departamento de Ingeniería
Informática, área de Lenguajes y Sistemas Informáticos.

\noindent Expone:

\noindent Que el alumno D. \nombre, con dni , ha realizado el Trabajo final de Grado en Ingeniería Informática titulado título de TFG. 

\noindent Y que dicho trabajo ha sido realizado por el alumno bajo la dirección del que suscribe, en virtud de lo cual se autoriza su presentación y defensa.

\begin{center} %\large
En Burgos, {\large \today}
\end{center}

\vfill\vfill\vfill

% Author and supervisor
\begin{minipage}{0.45\textwidth}
\begin{flushleft} %\large
Vº. Bº. del Tutor:\\[2cm]
Carlos López Nozal
\end{flushleft}
\end{minipage}
\hfill
\begin{minipage}{0.45\textwidth}
\begin{flushleft} %\large
Vº. Bº. del co-tutor:\\[2cm]
Raúl Marticorena Sánchez
\end{flushleft}
\end{minipage}
\hfill

\vfill

% para casos con solo un tutor comentar lo anterior
% y descomentar lo siguiente
%Vº. Bº. del Tutor:\\[2cm]
%D. nombre tutor


\newpage\null\thispagestyle{empty}\newpage




\frontmatter

% Abstract en castellano
\renewcommand*\abstractname{Resumen}
\begin{abstract}
En estos últimos años hemos sido testigos del crecimiento de la docencia en línea en múltiples instituciones académicas y empresas. Según estadísticas \cite{estadisticas-crecimiento-elearning} el elearning ha crecido un 900\% desde el año 2000. Las universidades, las empresas, los centros de formación vial, las academias y muchas más empresas e instituciones están optando por este tipo de educación, por ser más barata, más rápida y la preferida por los estudiantes. La docencia en línea es la nueva forma de educar y requiere de marcos de calidad que ayuden a conseguir los mejores resultados en la educación y la formación a través de internet.

En este proyecto se desarrolla una herramienta para medir la calidad de los cursos de Moodle. Para ello nos apoyamos en el estándar de calidad Quality Reference Framework for the quality of Moocs \cite{quality-reference-framework}. Además partimos de una versión anterior que funciona con las versiones 3.x de Moodle, por lo que una de las primeras ideas es hacer funcionar esta herramienta con versiones 4.x. Además, se han añadido nuevas reglas de calidad orientadas a validar la objetividad de la evaluación mediante cuestionarios. Las reglas se basan en aplicar umbrales en los índices de facilidad y discriminación, en la calificación aleatoria de las preguntas, en el índice validez interna y en el porcentaje de participación.

Para lograr este objetivo nos servimos de la API de Moodle \cite{moodle-api} que ofrece información suficiente para generar informes de calidad, además se utilizan ficheros estadísticos en formato JSON que proporcionan información útil sobre cuestionarios.
\end{abstract}

\renewcommand*\abstractname{Descriptores}
\begin{abstract}
moodle,calidad,springboot,maven,docker,cursos de moodle,framework de calidad, medidas psicométricas
\end{abstract}

\clearpage

% Abstract en inglés
\renewcommand*\abstractname{Abstract}
\begin{abstract}
In recent years we have witnessed the growth of online teaching in many academic institutions and companies.According to statistics \cite{estadisticas-crecimiento-elearning} e-learning has grown by 900\% since 2000.Universities, companies, road training centers, academies and many more companies and organizations are now using e-learning, academies and many more companies and institutions are opting for this type of education, as it is cheaper, faster and preferred by students. Online education is the new way of education and requires quality frameworks that help achieve the best results in education and formation through the Internet.

In this project we develop a tool to measure the quality of Moodle courses. For this purpose we rely on the quality standard Quality Reference Framework for the quality of Moocs \cite{quality-reference-framework}. In addition, we start from an earlier version that works with Moodle 3.x versions, so one of the first ideas is to make this tool work with 4.x versions. In addition, new quality rules have been added to validate the objectivity of the evaluation through questionnaires. The rules are based on applying thresholds on the ease and discrimination indexes, on the random scoring of questions, on the internal validity index and on the participation rate.

To achieve this goal we use the Moodle API \cite{moodle-api} which provides enough information to generate quality reports, and also statistical files in JSON format that provide useful information about quizzes.
\end{abstract}

\renewcommand*\abstractname{Keywords}
\begin{abstract}
moodle,quality,springboot,maven,docker,moodle courses,quality framework, psychometric measures
\end{abstract}

\clearpage

% Indices
\tableofcontents

\clearpage

\listoffigures

\clearpage

\listoftables
\clearpage

\mainmatter
\capitulo{1}{Introducción}

El e-learning ha sido en los últimos años una forma de educar a personas con horarios poco flexibles debido a su naturaleza principalmente asíncrona o sobrepasar otra clase de limitaciones de la enseñanza tradicional, sin embargo, debido a la situación de pandemia global y consecuente confinamiento, el e-learning ha tomado un papel principal en la educación\cite{muhammad2020hierarchical}.
La calidad de la enseñanza desde el punto de vista del alumno es el factor que más influye en la intención de ingreso y la propensión a recomendar la institución de enseñanza\cite{martinez2016perceived}.
Existen marcos de calidad que se podrían aplicar al e-learning, pero al no haber sido concebidos en concreto para este objetivo languidecen al hacerlo\cite{muhammad2020hierarchical}.
También existen trabajos que ponderan la factibilidad de implantar un sistema automático de evaluación de la calidad del e-learning sin entrar en detalles de los marcos de calidad a utilizar\cite{doneva2015automated} y otros que analizan los datos de forma menos automática para aplicar mejoras en el e-learning a posteriori\cite{ueda2017data} o aquellos que tienen una forma de mostrar los datos para la toma de decisiones pero obtienen la información por medio de entrevistas y encuestas\cite{mejia2020dashboard}.
Además, hay montones de modelos, estándares y marcos de calidad creados con el e-learning en mente pero se encuentran a niveles de abstracción demasiado altos o utilizan comprobaciones demasiado complicadas para automatizarlas a fecha de hoy.
El objetivo de este proyecto es crear una herramienta que pueda recoger información de forma automática sobre la calidad de los cursos en la plataforma de e-learning Moodle, que de hecho es utilizada por la Universidad de Burgos, para luego mostrar dicha información y así permitir a los docentes entrar en un ciclo de mejora de la calidad de sus cursos.

\capitulo{2}{Objetivos del proyecto}

Los objetivos del proyecto se pueden dividir en dos. Los objetivos funcionales, entendidos como aquellos que comprenden las funcionalidades que deseamos que tenga la aplicación desarrollada, y los objetivos técnicos, entendidos como las características técnicas que debe tener la aplicación.

\section{Objetivos funcionales}
En cuanto a los objetivos funcionales, se quiere que la aplicación permita a las personas que tengan permisos de gestión en cursos de Moodle, poder ver un informe detallado sobre las distintas fases del curso, entre las que encontramos: 
\begin{itemize}
    \item Diseño
    \item Implementación
    \item Realización
    \item Evaluación
\end{itemize}

 Cada fase esta compuesta por un conjunto de reglas de calidad. La definición de las reglas de calidad se formaliza calculando métricas/consultas sobre el LMS y comparándolo con unos valores umbrales. Los valores umbrales de las medidas son referencias documentadas y pueden ser configurables por el usuario. La evaluación de calidad de cada fase de un curso en línea se corresponde con el  porcentaje de  medidas/consultas sobre el curso cuyos valores están de dentro de los umbrales. Adicionalmente, el informe de calidad generado se debe poder exporta a un archivo Excel y se debe guardar un registro de los informes de calidad generados en cada curso en un archivo .csv en local.

Por otro lado, se desea que la aplicación sugiera al usuario cambios a realizar para mejorar, estos cambios se basan en la identificación de recursos o actividades del curso que no cumplen las reglas de calidad. Además, se mostrara una gráfica de evolución temporal de la calidad del curso, utilizando los registros de todas las evaluaciones del curso realizadas. Las reglas también se pueden agrupar en otras dos posibles , la categoría de roles que se clasifica en diseñador, facilitador, proveedor y la categoría de perspectivas que se clasifica en pedagógica, tecnológica, estratégica

Finalmente los objetivos se pueden enumerar como los siguientes:
\begin{enumerate}
    \item Mantenimiento de la aplicación, entendido como, actualización de dependencias, solución a errores.
    \item Compatibilizar la aplicación con nuevas versiones de Moodle.
    \item Añadir nuevas reglas relativas a cuestionarios
    \item Añadir exportación de informes de calidad a un archivo Excel.
    \item Disponer de un manual de usuario para la orientación en la aplicación.
    \item Apartado de contacto y ``about'' de la página.
\end{enumerate}

\section{Objetivos técnicos}
En cuanto a los objetivos técnicos, se desea que el proyecto siga un desarrollo basado en metodologías ágiles, dividiendo los avances en sprints, en los que se realizan una serie de tareas y se hacen reuniones en la fecha límite en cada uno para ver presentar los avances y recibir retrospectivas. Para seguir este proceso se utilizan herramientas como Zube y las issues de GitHub que permiten que todos los componentes del equipo puedan monitorizar los cambios y avances que se estén realizando. 

Paralelamente, es conveniente que la aplicación este basada en tecnologías escalables y fácilmente mantenibles. Para esto, se utiliza SpringBoot con Maven, que proporciona la capacidad de desarrollar y desplegar una aplicación rápidamente, además de proporcionar una configuración automática y las ventajas de la inyección de dependencias. Adicionalmente, queremos utilizar patrones de diseño para resolver los problemas que puedan surgir desarrollando la aplicación, siguiendo así unas buenas prácticas.
\capitulo{3}{Conceptos teóricos}


\section{Definiciones}

\subsection{E-learning}

El e-learning es la enseñanza impartida por medios electrónicos como internet, plataformas virtuales, medios audiovisuales…etc.

\subsection{Moodle}

Moodle es una plataforma de aprendizaje que permite a los profesores crear entornos de aprendizaje altamente personalizables. Fue creado por Martin Dougiamas que publicó su primera versión el 20 de agosto de 2002.

\subsection{Calidad}

La calidad se puede definir como el conjunto de características de un producto que satisfacen las expectativas del cliente.

\subsection{Web scraping}

Es el conjunto de técnicas utilizadas para extraer y almacenar información  de la web.

\section{Consultas}

\capitulo{4}{Técnicas y herramientas}
Para dar inicio a este capitulo, hay que entender la arquitecturadel proyecto. Para ello hay que pensar en como debe funcionar esta aplicación y como interactúa con el usuario.

eLearningQA interactúa con una API del servicio web de Moodle que le proporciona la mayoría de los datos de un curso de Moodle, esos datos se procesan para obtener resultados ilustrativos de la calidad de los cursos de Moodle, que se presentan en informes. 

\section{Modelo-Vista-Controlador}
La estructura de este proyecto sigue el patrón Modelo-Vista-Controlador\cite{mvc}. Este patrón arquitectónico divide el software en tres componentes:
\begin{itemize}
    \item \textbf{Modelo:} En este componentes se encuentran los datos y toda la lógica relativa a su obtención.
    \item \textbf{Vista:} En la vista se encuentra la interfaz gráfica y todos aquellos componentes que interactúen directamente con el usuario.
    \item \textbf{Controlador:} Este componente es el intermediario entre el modelo y la vista, este es el que gestiona las peticiones que se reciben de la vista y decide que datos mostrar y como, es decir, es el que contiene la lógica de negocio.
\end{itemize}

\begin{figure}[H]
    \centering
    \includegraphics[width=0.80\linewidth]{img/MVC.png}
    \caption{Pátrón de arquitectura MVC}
    \label{fig:mvc-imagen}
\end{figure}

Con el patrón MVC \ref{fig:mvc-imagen} se pueden construir páginas web escalables mantenibles y fáciles de expandir. Además es una opción atractiva dado su facilidad de implementación y su clara divisón entre Frontend y Backend, lo que facilita la separación de preocupaciones.

\section{Técnicas utilizadas en el backend}
Una vez entendida la arquitectura del proyecto se puede profundizar a cada una de las partes que lo componen.

Empezando con el Backend, esta aplicación esta programada en Java utilizando Spring como framework y Maven para el control de dependencias. Además cuenta con contenerización con Docker.

\subsection{SpringBoot y Maven}
Spring Framework proporciona un modelo de configuración y programación intuitivo y fácil de usar. El objetivo de Spring Framework\cite{spring-framework} es realizar todas las configuraciones internas necesarias para que los equipos de desarrollo solo tengan que encargarse de desarrollar la lógica de negocio y no dediquen excesivo tiempo a la configuración del proyecto.

Uno de los proyectos que ofrece Spring es SpringBoot\cite{spring-boot} que permite crear stand-alone y ejecutarlas rápidamente. Además, cuenta con servidores web embebidos lo que facilita enormemente el desarrollo de una aplicación web.

Para proveer de SpringBoot y de todas las dependencias que necesita esta aplicación se utiliza Maven que proporciona un fichero xml llamado POM que permite añadir todas las dependencias necesarias al proyecto fácilmente.

\subsection{Docker}
Docker \cite{docker} es una plataforma de desarrollo que permite desacoplar las aplicaciones de la infraestructura de forma que se puede enviar y desplegar aplicaciones rápidamente. Docker se basa en la contenerización \cite{contenerización}. 

La contenerización consiste en que las aplicaciones son empaquetadas con aquellas bibliotecas del sistema operativo necesarias para que se ejecute, de esta forma se pueden desplegar aplicaciones en cualquier sitio de forma rápida y ligera.

\begin{figure}[H]
    \centering
    \includegraphics[width=0.80\linewidth]{img/docker-vs-vm.png}
    \caption{Contenerización con Docker vs virtualización}
    \label{fig:docker-imagen}
\end{figure}

En la imagen anterior \ref{fig:docker-imagen}, se puede ver claramente la diferencia de los contenedores con las máquinas virtuales. Se puede ver que con los contenedores se ejecutan sobre la misma ``maquina'' compartiendo sistema operativo, mientras que con las máquinas virtuales cada aplicación se ejecuta en un sistema operativo.

Docker se ha implementado al proyecto \cite{dockerizar-entorno} utilizando principalmente el .war que se genera al compilar la aplicación. Con este .war, que lleva embebido un servidor Tomcat, se puede crear una Dockerfile que ejecute el .war compilado con el siguiente comando:
    \begin{verbatim}
        java -jar prototipo-0.4-SNAPSHOT.war
    \end{verbatim}
Con esto, se obtiene una imagen que debe ser ejecutada en un contenedor. 
\section{Técnicas utilizadas en el frontend}
El frontend se entiende como todos los componentes que interactúan directamente con el usuario y la lógica en ellos. Actualmente, existe una gran variedad de opciónes para implementar el frontend de una aplicación, tales como, Angular \cite{angular}, React \cite{react} o Vue.js \cite{vue}.

En el caso de eLearningQA, se utiliza JSP (Java Server Pages) \cite{jsp}. Las páginas JSP \ref{fig:diagrama-jsp}  son una mezcla de código HTML, XML y Java que son convertidas a un servlet el cual genera la vista. Existen opciones más sencillas de implementar y con más capacidades como las nombradas anteriormente. Sin embargo, para hacer interfaces básicas y rápidamente es una buena opción.

\begin{figure}[H]
    \centering
    \includegraphics[width=0.65\linewidth]{img/servlet.jpg}
    \caption{Funcionamiento del Servlet y JSP}
    \label{fig:diagrama-jsp}
\end{figure}
    
\section{Modelo de desarrollo software}
El modelo de desarrollo software es un enfoque sistemático que describe todas las actividades del ciclo de vida del desarrollo software. Existen varios enfoques \cite{modelo-desarrollo} pero el que se ha aplicado en el desarrollo de este proyecto ha sido el modelo ágil.

\subsection{Modelo ágil}
Este modelo se basa en la adaptabilidad, la colaboración y el desarrollo incremental. Además con este modelo fomenta la entrega rápida del software cosa muy valorada en este proyecto.

El modelo ágil consiste en la división del trabajo en iteraciones cortas llamadas sprint. Antes de cada sprint se asigna una cantidad de trabajo a cada participante en orden de prioridad. Cuando comienza el sprint se realiza el trabajo y se realizan reuniones diarias con el equipo para hacer un seguimiento de los avances. A medida que se terminan funcionalidades se integran y se realizan pruebas. Cuando termina el sprint se realiza una revisión de todas las tareas completadas y todos los miembros del equipo hacen sus comentarios sobre el sprint. 

Para aplicar esta metodología existen herramientas como Jira o Zube, que facilitan el trabajo del project manger, que sería el encargado de la gestión del equipo. En el caso de este proyecto, se ha utilizado Zube.io para aplicar la metodología ágil, de forma que los miembros del equipo podían subir sus avances y cuestiones, además de monitorizar sus avances con herramientas analíticas que ofrece Zube.io.

\begin{figure}[H]
    \centering
    \includegraphics[width=1\linewidth]{sprint-burndown.png}
    \caption{Analítica de burndown de un sprint del proyecto}
    \label{fig:enter-label}
\end{figure}

\subsection{Repositorio de proyecto y sistemas de control de versiones}
Para hacer un desarrollo sostenible y seguro es conveniente contar con un controlador de versiones, que no solo asegura que exista una copia de un proyecto en algún lugar, si no que permite a los miembros de un equipo trabajar en paralelo de forma rápida y cómoda. Además permite implementar CI/CD, que hace que el desarrollo y la integración sea rápida y segura.

La elección para el controlador de versiones de este proyecto es GitHub, que gracias a su facilidad de uso y a las posibilidades que ofrece facilita y mejora el desarrollo de aplicaciones.

Además, Github ofrece la posibilidad de llevar a cabo CI/CD \cite{ci-cd} con las GitHub Actions. La integración continua y la entrega continua son fases por las que pasan los productos software para hacer una entrega de producto continuo y sin errores. En este proyecto se utiliza el Java CI con Maven, esto se utiliza para hacer una complicación con Maven en cada commit, si la compilación ha ido correctamente el desarrollador podrá seguir con sus tareas, sin embargo, si ha habido errores en esa compilación sera necesario tomar medidas correctivas para que el proyecto no tenga errores en remoto.

\begin{figure}[H]
    \centering
    \includegraphics[width=1\linewidth]{img/ci-cd.png}
    \caption{Esquema ilustrativo de CI/CD}
    \label{fig:enter-label}
\end{figure}

Como conclusión, también es interesante añadir que con GitHub es posible conectar proyectos a herramientas de calidad como SonarCloud que pueden ayudar al mantenimiento de calidad del proyecto.

\subsection{SonarCloud}
SonarCloud \cite{sonar-cloud} es una herramienta que proporciona un amplio análisis de código y que permite a los desarrolladores conocer datos importantes sobre el código que están desarrollando como cobertura de test, brechas de seguridad, complejidad visual, duplicación de código, etc.

En este proyecto se utiliza SonarCloud para hacer un desarrollo sostenible, seguro y consistente. De forma que todas las personas que aporten en el desarrollo del código de la aplicación hagan un trabajo pulcro y consistente con el trabajo de los demás. SonarCloud, además, ayuda a evitar posibles brechas en la seguridad que pondrían en una mala posición tanto a desarrolladores como a clientes. El sonarQube de este proyecto se encuentra en



\capitulo{5}{Aspectos relevantes del desarrollo del proyecto}
\section{Ciclo de vida}
La realización de este trabajo se ha llevado a cabo en sprints de una duración de 14 días con reuniones entre sprint y sprint, y con frecuencia han habido reuniones a mitad de estos.

El proyecto empezó sintetizando una lista de aspectos a comprobar en los cursos Moodle a partir del marco de referencia de calidad del e-learning de MOOQ \cite{stracke2018quality} y un documento interno de UBUCEV proporcionado por los tutores.
Después tuve que decidir si crear una aplicación de escritorio o una aplicación web y qué framework utilizar para desarrollar la aplicación.
A partir de ahí, se hizo un prototipo para comprobar que era capaz de acceder a los \textit{web services} de \textit{Moodle} desde mi aplicación. Este prototipo solo mostraba la lista de cursos accedidos recientemente a partir de unas credenciales para la página de demostración de Moodle llamada Mount Orange School.

Más tarde establecí el ciclo de integración continua/despliegue continuo y lancé la aplicación en Heroku. Una semana después, al final del sprint había creado una versión del informe específico que solo realizaba una comprobación sobre los cursos. A partir de ahí, debido a las fechas (segunda quincena de diciembre) no hubo ninguna reunión con los tutores hasta la vuelta de las vacaciones de navidad. Durante ese periodo, añadí el resto de las comprobaciones al informe y me dediqué a completar partes de la memoria. Decidimos retrasar la entrega del trabajo al segundo semestre.

El siguiente paso fue añadir \textit{SonarCloud} para analizar el código, y arreglar la gran mayoría de los errores encontrados. Se añadió \textit{Bootstrap} para mejorar bastante el estilo de la página. Se añadieron varias mejoras de características no funcionales a la aplicación: un logo, configuraciones intercambiables, alertas de las causas de los fallos de las comprobaciones, página de error y gráficos de evolución de la calidad de los cursos. El resto del desarrollo se dedicó a solucionar errores pequeños.

\section{Proceso de obtención de llamadas a los servicios web}
Para obtener la información necesaria para las comprobaciones sobre los cursos de Moodle he tenido que realizar llamadas REST a distintas funciones de la API de servicios web de Moodle \cite{wsapifunctions-2021}. Sin embargo, la tabla que detalla la lista de funciones de la API omite los parámetros necesarios para las llamadas a las funciones.
Para averiguar qué parametros debía utilizar en cada función tuve que acceder al repositorio de Moodle \cite{moodlerepository-2022} y encontrar las funciones que detallaban el funcionamiento de la función de la API en la que estaba interesado cada vez.

El nombre de una función se puede dividir en dos partes: el nombre del componente que posee la función, y el nombre de la función. Las funciones están escritas en PHP y aparecen dentro de los archivos llamados ``externallib.php''. Para cada función existen tres funciones asociadas: 
\begin{itemize}
	\item <NOMBRE DE LA FUNCIÓN>
	\item <NOMBRE DE LA FUNCIÓN>\_parameters
	\item <NOMBRE DE LA FUNCIÓN>\_returns
\end{itemize}
La función a secas define el comportamiento de la función, y en su declaración se puede ver qué parámetros son opcionales y cuáles no.
La función con ``parameters'' al final describe los nombres y los tipos de datos de los parámetros que espera recibir la función.
La función con ``returns'' al final describe los nombres y los tipos de datos de los atributos que devuelve la función.
Por ejemplo, si quiero averiguar como llamar a la función mod\_forum\_get\_forums\_by\_courses
tengo acceder al fichero /mod/forum/externallib.php (supuesto por el nombre del componente que contiene la función, ``mod\_forum'' en este caso) y buscar las funciones get\_forums\_by\_courses, get\_forums\_by\_courses\_parameters, y get\_forums\_by\_courses\_returns.

Para probar el funcionamiento de las llamadas y al mismo tiempo obtener las clases que debía definir como POJO para manejar los datos recibidos como JSON (JavaScript Object Notation), es una forma de representar un objeto en formato de texto muy usado para transmitir información en aplicaciones web, hice lo siguiente:
Uso la página de demostración de Moodle llamada Mount Orange School para hacer pruebas.
Hago llamadas REST de forma manual con el navegador Chrome y obtengo el token con las credenciales de profesor (usuario:``teacher'' contraseña:``moodle'').\imagen{Token.png}{Obtención del token} Después utilizo el token para la siguiente llamada y obtengo una respuesta en formato JSON. En este caso la llamada solo necesita el nombre de la función, el formato de respuesta, y el token, pero en la mayoría de casos se requieren parámetros como identificadores de foro o de actividad. \imagen{JSON.png}{Obtención del JSON} Copio la respuesta en un conversor de JSON a POJO.\imagen{Conversor.png}{Uso del conversor} Creo las clases correspondientes, añado un constructor vacío y encapsulo los atributos. 


\section{Integración continua y despliegue continuo}
La integración continua consiste en la automatización de la compilación y ejecución de pruebas cada vez que se suben cambios al repositorio.
El despliegue continuo es la automatización del despliegue de un producto tras cada cambio en el repositorio.
He implementado la integración continua del proyecto con GitHub Actions, primero, establecí en el archivo pom.xml que la compilación del proyecto fuera en formato WAR (Web Application Resource), luego, creé el archivo ``maven.yml'' en la carpeta de workflows para establecer que cada vez que se realice un push en la rama principal el proyecto se compile y se ejecuten los tests con Maven.

He implementado el despliegue continuo en Heroku, la mayoría del proceso ha sido bastante intuitiva, ya que una de las opciones que ofrecía era GitHub como método de despliegue pero debido a que mi repositorio contiene por un lado la memoria y por otro el proyecto software, he tenido que añadir un buildpack (conjuntos de scripts de código abierto usados para compilar las aplicaciones en Heroku) que permite especificar una subcarpeta del repositorio para usarla como directorio raíz del proyecto software. También establecí en las opciones que el despliegue automático espere a que se supere la integración continua. También, tras definir una quality gate en SonarCloud, ha pasado a formar parte de la integración continua, y, en caso de no cumplirse los estándares de la quality gate (figura \ref{fig:qualitygate}), la integración continua cuenta como fallida.

\imagen{QualityGate.png}{\label{fig:qualitygate}Quality gate usada en el proyecto}

\section{Fallo de seguridad Heroku/TravisCI/Github}
El 9 de abril de 2022 unos repositorios privados de Github fueron descargados de forma indebida debido a un ataque por medio de Heroku y TravisCI. El atacante robó tokens de autorización de Github almacenados por Heroku y TravisCI el día 7 y los utilizó para buscar repositorios basándose en las organizaciones a las que pertenecían el día 8 para acabar descargándolos el día 9. Github descubrió esto el 12 de abril y avisó a Heroku el día siguiente \cite{githubsecurity-2022}. Tras tres días de investigación, Heroku revocó los tokens el 16 de abril, desabilitando el despliegue automático en el proceso y afirmó no reestablecerlo hasta cerciorarse de que hacerlo sea seguro. Heroku reestableció el servicio el 25 de mayo \cite{herokusecurity-2022}.

\section{Uso del nombre Moodle}
Como el objetivo inicial del proyecto es integrar Moodle en la aplicación, el nombre de tanto el proyecto como la aplicación iba a ser ``MoodleQA'', pero en medio del desarrollo los tutores y yo descubrimos que al ser ``Moodle'' una marca registrada \cite{moodletrademark-2022}, existe cierta cantidad de restricciones sobre el uso de esta palabra. Las restricciones que nos conciernen son no poder utilizar ``Moodle'' en el nombre de tu software ni en el nombre de dominio ni usarlo para describir tu aplicación de forma que los usuarios crean que estás asociado con Moodle si no lo estás. Por ello he tenido que revisar todas las menciones que hago a Moodle tanto en la memoria y anexos como en la aplicación.

\section{Informes: pestañas vs breadcrumbs}
A la hora de mostrar los informes se presenta la disyuntiva de si hacerlo en una sola pestaña de navegador y permitir al usuario navegar entre los distintos cursos con ayuda de una miga de pan o breadcrumb, o por el contrario generar cada informe en una nueva pestaña. Me he decantado por generar una pestaña por informe por dos razones: el rendimiento y la comparación de informes. Cada vez que se genera un informe se descarga la información necesaria para las consultas y se realizan las consultas sobre dicha información, para un servidor puede que la carga no sea mucha y se puedan generar los informes al instante, pero ejecutando la aplicación en local desde mi ordenador portátil he llegado a tener esperas de alrededor de diez segundos por un solo informe. A la hora de comparar informes es más ágil el cambiar de pestaña que retroceder a la página anterior e ir al otro informe.

\section{SonarCloud}
En esta sección agrupo y expongo dos sucesos interesantes ocurridos relacionados con \textit{SonarCloud} tras la implantación de esta herramienta al repositorio.
\subsection{Definición del quality gate de SonarCloud}
En SonarCloud la quality gate es una serie de condiciones sobre la calidad del codigo, por ejemplo, que la puntuación de seguridad no esté por debajo de ``A''. Existe una quality gate predefinida llamada Sonar way, pero está definida para proyectos reales, así que en mi caso he decidido definir mi propia quality gate a partir de otros proyectos en Sonarcloud de la misma índole. Realicé una búsqueda y encontré 25 TFGs de la UBU, y calculé las medianas de dichos trabajos para obtener las condiciones de mi quality gate.

\subsection{SonarCloud en el ciclo de desarrollo}
Desde que añadí \textit{SonarCloud} al ciclo de integración y despliegue continuo he creado el hábito de comprobar los bugs, smells, y vulnerabilidades generados tras cada commit, esto evita que la cantidad de problemas se acumule y a su vez hace que el código sea más fácil de mantener. Además, el hecho de entender los code smells conforme los solucionaba me ha enseñado a tenerlos con menor frecuencia. A continuación se muestra un gráfico generado por SonarCloud del número de problemas a lo largo del tiempo en el que se aprecia que tras solucionar los problemas iniciales el gráfico se mantiene relativamente plano. \imagen{GraficoSonarCloud.png}{\label{fig:sonarcloud}Gráfico del número de problemas a lo largo del tiempo}

\section{Diseño del logo de la aplicación}
Debido a que la aplicación realiza una serie de consultas se me ocurrió que el logo lo formasen unas aspas y un check, pero sin el uso de colores el símbolo resultante puede resultar confuso, así que rodeé la parte del check con color verde y la parte de las aspas con color rojo para hacerlo más entendible.
Los colores usados son los de Bootstrap para crear concordancia con el resto de la página y de paso crear las imágenes de check y aspas usados en los resultados de las consultas y que de esta forma se mimeticen en el color de fondo de la celda. El texto del logo es blanco con reborde negro para hacerlo legible con independencia del color del fondo. \imagen{FullLogo.png}{Logo de la aplicación}

\section{Almacenamiento de registros de informes}
La aplicación almacena los registros de los informes generados para poder generar los gráficos de evolución. Ejecutando la aplicación en local no supone ningún problema, sin embargo, el almacenamiento en Heroku es volátil \cite{ephemeralHeroku-2018}. La solución a esto sería contratar un proveedor de active storage como Amazon S3 para tener un almacenamiento que funcione independiente a la aplicación \cite{activeStorageOverview-2019}.

\section{Redefinición de regla de comentarios del profesor}
Originalmente lo que se comprobaba era si el ratio entre items de calificación comentados dividido entre el número total de items comentables se encontraba por encima de cierto umbral, sin embargo, esto da falsos negativos en caso de que los estudiantes no realicen las entregas correspondientes. Si reducimos el concepto de items comentables a aquellos que tengan calificación evitaremos este problema.

También existe otro problema, actualmente, para poder realizar la consulta se comprueban los comentarios de la columna de comentarios de los calificadores de todos los alumnos, columna que el profesor puede elegir no mostrar en el calificador. Este problema causa que nuestra regla compruebe de forma implícita una decisión de diseño aparte de su objetivo principal, que es comprobar que el profesor dé retroalimentación a los alumnos al corregir (que corresponde a la fase de realización).
\capitulo{6}{Trabajos relacionados}

En este apartado se recogen todas las aplicaciones y proyectos que ofrecen funcionalidades parecidas a las que se implementan en eLearningQA. Como en un estudio de mercado se comprueba que ofrecen otras personas o entidades para identificar las necesidades que se pueden cubrir. 

\section{Moodle Course Checker}
Course Checker \cite{moodle-course-checker} es un plugin de Moodle que ofrece la posibilidad de hacer hasta diez comprobaciones comparativas entre cursos propias de la fase de implementación. Este plugin permite corregir fallos en la configuración de cursos con el fin de mantener una consistencia interna entre los distintos cursos.

\section{Course Check Blocks}
Course Check Blocks \cite{course-checks-blocks} es un plugin que ofrece 11 comprobaciones automáticas que permiten al usuario mantener un mínimo de calidad en el curso y aumentar las oportunidades de mejora. Además ofrece una funcionalidad adicional que permite borrar cualquier bloque vacío del curso.

\section{Dashboard for Evaluating the Quality of Open Learning Courses}
Un artículo \cite{modelo-sustanabilty} publicado en la revista Sustainbility por Gina Mejía-Madrid, Faraón Llorens-Largo, y Rafael Molina-Carmona en 2020, presenta un modelo para la evaluación de calidad en cursos, así como, un panel visual que permite ver los resultados obtenidos y compararlos de forma rápida y sencilla.

\section{A Hierarchical Model to Evaluate the Quality of Web-Based E-Learning Systems}
En un estudio publicado en la revista Sustainability por Abdul Hafeez Muhammad y otros miembros de varias universidades \cite{marco-calidad-muhammad}, muestra un modelo que determina cuales son los factores más relevantes a tener en cuenta para mantener un curso con calidad. Para la obtención de este modelo se encuestaron a 157 sujetos para obtener los criterios más importantes para la calidad de un curso, una vez obtenidos esos criterios se les proporcionó a 51 sujetos esos criterios para compararlos entre sí para obtener la importancia de cada uno de ellos.

Así pues, el modelo obtenido \ref{fig:marco-muhammad} se puede utilizar como marco para el desarrollo de cursos de calidad, es interesante tener en cuenta este trabajo por que, como otros modelos, también proporciona un marco para el diseño de cursos en línea de calidad.

\begin{figure}[H]
    \centering
    \includegraphics[width=1\linewidth]{img/modelo-evaluacion-calidad.png}
    \label{fig:marco-muhammad}
    \caption{Marco de calidad definido en ``Hierarchical Model to Evaluate the Quality of Web-based E-Learning Systems''}
\end{figure}


\section{Comparación entre trabajos relacionados y eLearningQA}
Finalmente, con todas las opciones que se han visto, se puede hacer una comparativa con eLeaarningQA.

\begin{table}[H]
\resizebox{\textwidth}{!}{%
	\begin{tabular}{l|ccc}
		\hline
		\rowcolor[HTML]{FFFFFF} 
		\textbf{Característica} & \textbf{eLearningQA} & \textbf{Course Checker} & \textbf{
            \begin{tabular}[c]{@{}c@{}}
                Course Checks\\ Block
            \end{tabular}}         
        \\ \hline
        
		\rowcolor[HTML]{EFEFEF} 
		Idiomas & Español & \begin{tabular}[c]{@{}c@{}}Español, inglés,\\ alemán, y portugués\end{tabular} & \begin{tabular}[c]{@{}c@{}}Español, inglés,\\ portugués, y griego\end{tabular} \\
		Nº de comprobaciones &23 & 10 & 8\\
		\rowcolor[HTML]{EFEFEF} 
		\begin{tabular}[c]{@{}l@{}}Ejecución independiente\\ de comprobaciones\end{tabular} & No & Sí & No\\
		\begin{tabular}[c]{@{}l@{}}Contiene enlaces para\\ solventar los problemas\end{tabular} & Sí & Sí & No\\
		\rowcolor[HTML]{EFEFEF} 
		\begin{tabular}[c]{@{}l@{}}Versiones de Moodle\\ compatibles\end{tabular}& v3.8+ & v3.6+ & v2.6+\\
		Tipo & Aplicación web & Plugin & Plugin\\ \hline
	\end{tabular}%
}
\end{table}






\capitulo{7}{Conclusiones y Líneas de trabajo futuras}

\section{Líneas de trabajo futuras}
\begin{itemize}
	\item
	El método que utiliza la aplicación para realizar la comprobación de si el profesor responde a las dudas de los alumnos es una solución preliminar e incompleta. Se tiene como objetivo a futuro encontrar una forma más fiable de determinar qué es una duda y cuándo ha sido resuelta.
	El uso de modelos basados en el procesamiento del lenguaje natural puede ser un campo exploratorio que permita poder clasificar un mensaje del foro como una respuesta de dudas de un profesor. Pensamos que el diseño experimental y el posterior análisis de un clasificador con este cometido es suficientemente complejo para ser considerado un TFG por sí mismo.
	\item
	De las cinco consultas definidas referentes a los cuestionarios Moodle, no hay ninguna implementada en la aplicación, y esto se debe a que al menos cuatro de ellas no están soportadas por la API de servicios web de Moodle.
	Una posible solución a estas comprobaciones es el uso de técnicas de web scraping, pero teniendo en cuenta que las páginas que contengan la información deseada pueden variar en función del servidor que aloje la plataforma, no es una solución trivial.
\end{itemize}


\bibliographystyle{plain}
\bibliography{bibliografia}

\end{document}
