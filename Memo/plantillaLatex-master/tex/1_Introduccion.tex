\capitulo{1}{Introducción}

eLearningQA es una herramienta que trabaja sobre Moodle y que permite a sus usuarios visualizar informes
sobre sus cursos de Moodle. Esta herramienta proporciona un valor importante para los diseñadores de
los cursos, para profesores que administren, o para cualquier gestor de cursos en general. Para el 
desarrollo correcto de esta aplicación, tendremos que investigar sobre la calidad de cursos online masivos, abiertos, gratuitos (MOOC). 

Por otro lado, aplicaremos los conocimientos adquiridos durante el transcurso académico por la Universidad de Burgos, para hacer un proceso de desarrollo acorde a la buenas prácticas de desarrollo software, a las metodologías ágiles y otros conceptos importantes.

En este documento, explicaremos todos los aspectos relacionados con el proyecto, tanto la parte práctica como la teórica. Mostraremos el proceso de desarrollo, los aspectos técnicos y las contingencias encontradas durante el mismo, una vez realizado el proyecto daremos conclusiones con los aspectos importantes a mencionar. 
