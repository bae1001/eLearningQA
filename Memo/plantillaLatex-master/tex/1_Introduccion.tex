\capitulo{1}{Introducción}

El e-learning ha sido en los últimos años una forma de educar a personas con horarios poco flexibles debido a su naturaleza principalmente asíncrona o sobrepasar otra clase de limitaciones de la enseñanza tradicional, sin embargo, debido a la situación de pandemia global y consecuente confinamiento, el e-learning ha tomado un papel principal en la educación\cite{muhammad2020hierarchical}.
La calidad de la enseñanza desde el punto de vista del alumno es el factor que más influye en la intención de ingreso y la propensión a recomendar la institución de enseñanza\cite{martinez2016perceived}.
Existen marcos de calidad que se podrían aplicar al e-learning, pero al no haber sido concebidos en concreto para este objetivo languidecen al hacerlo\cite{muhammad2020hierarchical}.
También existen trabajos que ponderan la factibilidad de implantar un sistema automático de evaluación de la calidad del e-learning sin entrar en detalles de los marcos de calidad a utilizar\cite{doneva2015automated} y otros que analizan los datos de forma menos automática para aplicar mejoras en el e-learning a posteriori\cite{ueda2017data} o aquellos que tienen una forma de mostrar los datos para la toma de decisiones pero obtienen la información por medio de entrevistas y encuestas\cite{mejia2020dashboard}.
Además, hay montones de modelos, estándares y marcos de calidad creados con el e-learning en mente pero se encuentran a niveles de abstracción demasiado altos o utilizan comprobaciones demasiado complicadas para automatizarlas a fecha de hoy.
El objetivo de este proyecto es crear una herramienta que pueda recoger información de forma automática sobre la calidad de los cursos en la plataforma de e-learning Moodle, que de hecho es utilizada por la Universidad de Burgos, para luego mostrar dicha información y así permitir a los docentes entrar en un ciclo de mejora de la calidad de sus cursos.
