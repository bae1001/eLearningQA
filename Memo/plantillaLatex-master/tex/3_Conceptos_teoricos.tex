\capitulo{3}{Conceptos teóricos}


\section{Definiciones}

\subsection{E-learning}

El e-learning es la enseñanza impartida por medios electrónicos como internet, plataformas virtuales, medios audiovisuales…etc.

\subsection{Moodle}

Moodle es una plataforma de aprendizaje que permite a los profesores crear entornos de aprendizaje altamente personalizables. Fue creado por Martin Dougiamas que publicó su primera versión el 20 de agosto de 2002\cite{dougiamas2002interpretive}.

\subsection{Calidad}

La calidad se puede definir como la capacidad de satisfacer una serie de necesidades y en el caso del e-learning se trata de las necesidades educativas del alumno, como la calidad del material educativo o la ayuda a la comprensión de este.

\subsection{Web scraping}

Es el conjunto de técnicas utilizadas para extraer y almacenar información de la web, un programa que hace esto de forma automática se llama web crawler.

\subsection{Web services}

Es un medio de comunicación entre dos aplicaciones en ordenadores distintos dentro de una red mediante el uso de distintos protocolos. Se puede pedir cierta información o acciones al servidor por medio de llamadas a funciones. En el caso de los web services que proporciona Moodle existe un servidor que hace de intermediario entre cliente y proveedor\cite{moodle-2020}.

Ahora paso a explicar las tres dimensiones del marco de referencia de calidad del e-learning del MOOQ\cite{stracke2018quality}, en el que me baso:
\subsection{Fases}

\subsubsection{Análisis}
En esta fase se definen los objetivos, el contexto, y los recursos(docentes, tiempo, presupuesto...) para la ejecución para comprender la situación inicial.

\subsubsection{Diseño}
En esta fase se define lo que se planea hacer a partir de los resultados de la fase de análisis, como por ejemplo el enfoque de enseñanza que se piensa llevar a cabo.

\subsubsection{Implementación}
En esta fase se define de qué manera se van a llevar a cabo los planes descritos en la fase de diseño, como por ejemplo de qué manera se va a producir el contenido.

\subsubsection{Realización}
Esta fase es en la que se interactúa con el alumno, se gestionan los problemas técnicos y las dudas de los alumnos, además de evaluar su aprendizaje.

\subsubsection{Evaluación}
En esta última fase se evalúa la calidad del resto de fases mediante encuestas, entrevistas, u otros medios.

\subsection{Roles}

\subsubsection{Diseñadores}
Los encargados de decidir de qué forma se van a impartir el curso y generan el contenido(Autores, expertos en el tema, diseñadores instruccionales). 

\subsubsection{Facilitadores}
Son aquellos que conocen la materia a enseñar y son capaces de explicarlo y dar feedback, además de seguir el aprendizaje de los alumnos.

\subsubsection{Proveedores}
Son los encargados de proveer los medios digitales para llevar a cabo el e-learning(programadores, diseñadores y desarrolladores de software).

\subsection{Perspectivas}

\subsubsection{Pedagógica}
El punto de vista que se centra en el contenido y en el aprendizaje por parte del alumno. Los procesos relacionados tienen que ver con el contenido, el feedback, y la evaluación de los alumnos.

\subsubsection{Tecnológica}
El punto de vista que se centra en las necesidades tecnológicas del curso. La mayoría de los procesos está relacionada con esta perspectiva debido a la naturaleza del e-learning.

\subsubsection{Estratégica}
El punto de vista que se centra en la consecución de los objetivos del curso dentro del tiempo y presupuesto establecidos para este. Los procesos relacionados tienen que ver con los objetivos, conceptos y el contexto en el que se enseña(presupuesto, demanda, competencia...).


\section{Consultas}
