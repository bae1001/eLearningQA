\capitulo{5}{Aspectos relevantes del desarrollo del proyecto}
\section{Ciclo de vida}
La realización de este trabajo se ha llevado a cabo en sprints de una duración de 14 días con reuniones entre sprint y sprint, y con frecuencia han habido reuniones a mitad de estos.

El proyecto empezó sintetizando una lista de aspectos a comprobar en los cursos Moodle a partir del marco de referencia de calidad del e-learning de MOOQ \cite{stracke2018quality} y un documento interno de UBUCEV proporcionado por los tutores.
Después tuve que decidir si crear una aplicación de escritorio o una aplicación web y qué framework utilizar para desarrollar la aplicación.
A partir de ahí, se hizo un prototipo para comprobar que era capaz de acceder a los web services de Moodle desde mi aplicación.


