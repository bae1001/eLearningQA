\capitulo{2}{Objetivos del proyecto}

El objetivo principal de este trabajo es desarrollar una aplicación web
que permita al profesor evaluar las distintas fases de diseño instruccional de un curso de Moodle (diseño, implementación, realización, evaluación), tal como  recomiendan algunos frameworks internacionales de calidad en e-learning \cite{stracke2018quality}.

Los objetivos técnicos de este trabajo para conseguir el objetivo principal son:
\begin{enumerate}
	\item Definir las entidades y atributos de nuestro modelo de calidad a partir el modelo de datos de los cursos en Moodle: Usuarios, Roles, Cursos, Grupos, Actividades, Tareas, Recursos, Surveys, Feedbacks, Foros, Debates, y Comentarios.
	\item Conocer los servicios Web proporcionados por Moodle para interactuar con las entidades y atributos de nuestro modelo de calidad. 
	\item Definir un plan de calidad para aplicar a las entidades del modelo de cursos de Moddle a partir de frameworks internacionales de calidad en e-learning y recomendaciones del Centro de Enseñanza Virtual de la Universidad de Burgos (UBUCEV). 
	\item Definir evaluaciones del plan de calidad 
	personalizadas para distintos tipos de cursos: docencia reglada, tfg y comunidades y grupos.
	\item Diseñar indicadores cualitativos y cuantitativos de calidad de cada fase de diseño instruccional del curso en línea (diseño, implementación, realización y evaluación) 	
\end{enumerate}
