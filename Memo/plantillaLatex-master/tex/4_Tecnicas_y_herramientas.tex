\capitulo{4}{Técnicas y herramientas}

\begin{itemize}
	\item \textbf{Web scraping:}
	Es el conjunto de técnicas utilizadas para extraer y almacenar información de la web, un programa que hace esto de forma automática se llama web crawler. Algunas de estas técnicas nos permiten obtener información que no podríamos sacar con otras herramientas como por ejemplo si una página contiene enlaces fuera de Moodle a partir de su código HTML.
	\item \textbf{Web services:}
	Es un medio de comunicación entre dos aplicaciones en ordenadores distintos dentro de una red mediante el uso de distintos protocolos. Se puede pedir cierta información o acciones al servidor por medio de llamadas a funciones. En el caso de los web services que proporciona Moodle existe un servidor que hace de intermediario entre cliente y proveedor \cite{moodle-2020}. Por ejemplo, llamando a la función core\_grades\_get\_grades podríamos obtener las notas de un alumno.
	\item \textbf{Logs:}
	Son registros de actividad que Moodle crea a partir de los eventos que realizan los usuarios, como publicar en un foro o empezar un cuestionario. Se pueden descargar en forma de archivo. Un ejemplo del uso que le daríamos sería obtener la actividad de un profesor para saber si responde a las dudas de los alumnos en los foros.
\end{itemize}


