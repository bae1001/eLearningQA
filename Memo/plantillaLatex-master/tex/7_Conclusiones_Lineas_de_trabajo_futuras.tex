\capitulo{7}{Conclusiones y Líneas de trabajo futuras}
\section{Conclusiones}
	En el estudio de trabajos relacionados hemos visto la importancia de controlar la calidad en los procesos de enseñanza y aprendizaje de e-learning. La calidad en este ámbito suele tomar un papel secundario, pero no debería, ya que influye directamente en la lealtad del estudiante, cuestión que está adquiriendo relevancia en los últimos años.
	
	Los objetivos marcados para este proyecto fueron:
	\begin{enumerate}
		\item Definir las entidades y atributos de nuestro modelo de calidad a partir el modelo de datos de los cursos en Moodle: Usuarios, Roles, Cursos, Grupos, Actividades, Tareas, Recursos, Surveys, Feedbacks, Foros, Debates, y Comentarios.
		
		Este objetivo se ha conseguido, se han llegado a definir las clases necesarias para recibir la información pertinente para los análisis que realizamos utilizando una conversión a partir de sus representaciones en JSON.
		
		\item Conocer los servicios Web proporcionados por Moodle para interactuar con las entidades y atributos de nuestro modelo de calidad.
		
		Este objetivo también se ha conseguido, con complicaciones. Ha sido necesario acceder al código fuente de Moodle desde GitHub \cite{moodlerepository-2022} para rellenar las lagunas en la documentación oficial de su API de servicios Web \cite{wsapifunctions-2021}.
		
		\item Definir un plan de calidad para aplicar a las entidades del modelo de cursos de Moddle a partir de frameworks internacionales de calidad en e-learning y recomendaciones del Centro de Enseñanza Virtual de la Universidad de Burgos (UBUCEV). 
		
		Este objetivo se ha cumplido de forma satisfactoria, se han asociado responsabilidades, perspectivas y procesos a cada una de las comprobaciones que hemos definido en el plan de calidad (sección \ref{consultas}).
		
		\item Definir evaluaciones del plan de calidad 
		personalizadas para distintos tipos de cursos: docencia reglada, tfg y comunidades y grupos.
		
		Este objetivo se ha cumplido de forma parcial, se han definido archivos de configuración para cada tipo de curso, estas configuraciones controlan cómo de estrictas o laxas son algunas de las comprobaciones del plan de calidad, pero lo ideal sería que se pudieran activar o desactivar reglas según la configuración usada para hacerlo más personalizable.
		
		\item Diseñar indicadores cualitativos y cuantitativos de calidad de cada fase de diseño instruccional del curso en línea (diseño, implementación, realización y evaluación) 
		
		Este objetivo se ha cumplido, en el plan de calidad hemos definido implícitamente un indicador cualitativo para cada una de las comprobaciones realizadas, y, a partir de esos indicadores, otros que resumen el desempeño en cada una de las fases.
	\end{enumerate}
	
	Se ha desarrollado y desplegado una aplicación open source, plenamente funcional, que puede ser utilizada por profesores para recibir una evaluación de su diseño e implementación del curso en Moodle.

	Siendo este mi segundo trabajo de fin de grado, dispongo de un punto de referencia para ver qué diferencias hay entre ambos trabajos y sus consecuencias.
	
	La primera gran diferencia es la organización del desarrollo en sprints, careciendo el primero de una organización temporal definida. Esto ha hecho mucho más fácil marcarse objetivos y cumplirlos manteniendo un ritmo de trabajo constante.
	
	La presencia de dos tutores en oposición a uno solo permite tener más perspectivas sobre el trabajo a la hora de tener ideas sobre nuevas funcionalidades a implementar, sobretodo si están especializados en materias diferentes. También, cuando alguno de ellos no podía reunirse, el otro estaba disponible, cosa que resulto bastante conveniente a la hora de no perder el ritmo que llevaba. Por último, a menudo me han presentado herramientas para facilitar y agilizar el desarrollo del trabajo.
	
	La complejidad de este trabajo es bastante mayor y me ha obligado a aprender sobre algunas tecnologías con las que no estaba familiarizado como las herramientas de integración continua y despliegue continuo, cuestiones de desarrollo web (CSS, JavaScript, sesiones HTTP etc.), trabajar con una API de servicios web y adaptarse a ella. Este proceso de aprendizaje me ha resultado ameno.

\section{Líneas de trabajo futuras}
Aunque la aplicación desarrollada en este TFG es funcionalmente completa existen múltiples factores que hacen que pueda necesitar de futuras adaptaciones. Se considera interesante tener en cuenta factores como: la subjetividad intrínseca asociada a los procesos de calidad y las fuentes tecnológicas para definir controles de calidad automáticos, las diferencias en la distribución de responsabilidades de roles académicos, el tamaño de las instituciones académicas y las diferentes soluciones tecnológicas de implementación de cursos en línea .El análisis detallado de cada factor hace que surjan muchas líneas de trabajo futuras, a continuación se enumeran algunas que se han considerado interesantes.
\begin{itemize}
	\item
	El método que utiliza la aplicación para realizar la comprobación de si el profesor responde a las dudas de los alumnos es una solución preliminar e incompleta. Se tiene como objetivo a futuro encontrar una forma más fiable de determinar qué es una duda y cuándo ha sido resuelta.
	El uso de modelos basados en el procesamiento del lenguaje natural puede ser un campo exploratorio que permita poder clasificar un mensaje del foro como una respuesta de dudas de un profesor. Pensamos que el diseño experimental y el posterior análisis de un clasificador con este cometido es suficientemente complejo para ser considerado un TFG por sí mismo. Otro problema son las situaciones en las que la duda del alumno ha sido respondida por otro alumno y no hace falta responderla o se resuelve la duda en un comentario independiente en el mismo foro y no se detecte como respuesta.
	\item
	De las cinco consultas definidas referentes a los cuestionarios Moodle (sección \ref{consultas}), no hay ninguna implementada en la aplicación, y esto se debe a que al menos cuatro de ellas no están soportadas por la API de servicios web de Moodle.
	Una posible solución a estas comprobaciones es el uso de técnicas de web scraping, pero teniendo en cuenta que las páginas que contengan la información deseada pueden variar en función del servidor que aloje la plataforma, no es una solución trivial.
	\item
	Por el momento la aplicación solo integra cursos Moodle, pero sería conveniente que la aplicación permitiera analizar cursos online de otros LMS como Blackboard o Edmodo. Sin embargo, realizar los cambios para esto supondría adaptarse a las APIs de servicios correspondientes suponiendo que sean lo suficientemente parecidas, y en caso de no poder acceder a la información necesaria mediante servicios web, habría que implementar otras formas de acceder a la información necesaria.
	\item
	Los plugin de Moodle con los que comparo la aplicación tienen más idiomas disponibles, esto se debe a que están dispuestos de forma que cualquiera pueda aportar sus propias traducciones, sin embargo, el internacionalizar la aplicación la haría más competitiva.
	\item
	La lista de cursos que muestra la aplicación es la lista de los cursos en los que se encuentra matriculado el usuario con independencia de su rol o los permisos que tenga. Sería interesante seguir mostrando los mismos cursos pero sin resaltar en caso de que no se tengan los permisos necesarios para realizar las consultas para poder contactar al administrador en caso de problemas con los permisos.
	\item
	Sería muy interesante poder activar y desactivar las diferentes consultas desde los archivos de configuración si tenemos en cuenta que algunas consultas no aplican para algunos tipos de curso. Sin embargo, a la hora de calcular las estadísticas posteriores habría que adaptarse según las consultas que estén activadas, algo que es difícil teniendo en cuenta que la matriz Rol-Responsabilidad del informe se obtiene multiplicando un vector con los puntos obtenidos de cada consulta por una matriz que contiene la cantidad de puntos que supone el cumplimiento de cada comprobación para cada combinación de rol y perspectiva. El objetivo sería buscar otra forma de realizar el cálculo para facilitar la implementación de lo primero.
\end{itemize}