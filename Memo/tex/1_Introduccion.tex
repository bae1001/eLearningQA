\capitulo{1}{Introducción}

eLearningQA es una herramienta que trabaja sobre Moodle y que permite a los usuarios visualizar informes de calidad sobre sus cursos de Moodle. Esta herramienta proporciona un valor importante para los diseñadores de los cursos, para profesores que los administran, o para cualquier gestor de cursos en general. Para el desarrollo correcto de esta aplicación, se ha realizado una  investigación sobre la calidad de cursos en línea masivos y abiertos (MOOC).

Este proyecto está basado en estándares de calidad definidos con el fin de crear marcos de calidad para que profesores y diseñadores tengan una guía para desarrollar sus cursos. Con estos marcos se dispone de una guía para desarrollar una aplicación basada en conceptos estudiados y formalizados por profesionales de la enseñanza.

El objetivo de calidad de eLearningQA es conseguir el aprovechamiento máximo de los recursos a disposición de una institución educativa con el fin de que sus estudiantes y profesores hagan el mejor uso de los productos software de gestión de aprendizaje como Moodle.

Los documentos en los que se ha basado y han definido el progreso de la aplicación son: el estándar de calidad para MOOCs \cite{quality-reference-framework} que ofrece la base para hacer un desarrollo ordenado y razonado. Este marco define distintas dimensiones \ref{tabla:dimensiones} (fases, perspectivas y roles) que abarcan el proceso de la enseñanza a distancia desde que se que comienza el análisis para la creación de un curso hasta que el curso termina.

\tablaSmall{Tabla de dimensiones del marco de referencia de calidad}{l c}{dimensiones}
{ \multicolumn{1}{l}{Dimensión} & Contenido\\}{ 
Fases & Análisis, Diseño, Implementación, Realización y Evaluación\\
Perspectivas  & Pedagógica, Estratégica y Tecnológica\\
Roles & Diseñador, Facilitador, Proveedor\\
} 

Por otro lado, el estudio de modelos de calidad de elearning \cite{modelos-calidad-elearning} es otro documento que define claramente la calidad en la docencia en línea y da ejemplos de modelos de elearning de calidad. En este documento pueden ver los resultados positivos en docencia en línea de varias instituciones académicas y de los sistemas de educación de los que disponen para lograr esos resultados.

Este proyecto ha pasado por varias versiones anteriormente. Con las nuevas versiones de Moodle se han encontrado problemas de compatibilidad por lo que en esta nueva versión se ha habilitado la aplicación para que pueda funcionar con las versiones 4.x de Moodle, además se han solucionado problemas de compatibilidad con versiones anteriores a la 4.0 de Moodle y se ha aumentado funcionalidad, como la exportación de informes de calidad a un archivo Excel y nuevas reglas relativas a la calidad de cuestionarios..

En este documento, se explican todos los aspectos relacionados con el proyecto, tanto la parte práctica como la teórica. Mostraremos el proceso de desarrollo, los aspectos técnicos y las contingencias encontradas durante el mismo, además, se presentan conclusiones con los aspectos importantes a mencionar. 

\section{Open Learning}
Open learning \cite{open-learning} es un termino que se refiere a un tipo de educación en la que los estudiantes tienen flexibilidad para elegir el lugar, el marco temporal, el tipo de métodos de aprendizaje y otras características relacionadas con el proceso de aprendizaje. 

Así pues, una institución académica que ofrece open learning es aquella que ofrece a los estudiantes la oportunidad de aprender de distintas formas independientemente del tiempo o el espacio. 

Por ello, este modelo de aprendizaje se esta popularizando y se esta extendiendo cada vez más entre las personas, ya que permite que a estudiantes de todas las edades compaginar el aprendizaje con otras actividades. Y es por esto, que a las instituciones educativas están implementando este modelo de aprendizaje que no solo adapta el aprendizaje al estudiante sino que lo hace más alcanzable económicamente dada su diferencia de costo con el aprendizaje presencial tradicional.


