\capitulo{1}{Introducción}

eLearningQA es una herramienta que trabaja sobre Moodle y que permite a sus usuarios visualizar informes sobre sus cursos de Moodle. Esta herramienta proporciona un valor importante para los diseñadores de los cursos, para profesores que administren, o para cualquier gestor de cursos en general. Para el desarrollo correcto de esta aplicación, hemos investigado sobre la calidad de cursos en línea masivos y abiertos (MOOC).


Por otro lado, hemos aplicado los conocimientos adquiridos durante el transcurso del grado por la Universidad de Burgos, para hacer un proceso de desarrollo acorde a las buenas prácticas de desarrollo software, a las metodologías ágiles y otros conceptos importantes. 

Además, nos hemos basado en estándares de calidad definidos con el fin de crear marcos de calidad para que profesores y diseñadores 
tengan una guía para desarrollar sus cursos. Con estos marcos disponemos de una guía para desarrollar una aplicación basada en conceptos estudiados y formalizados por profesionales de la enseñanza.

El objetivo de calidad de eLearningQA es conseguir el aprovechamiento máximo de los recursos a disposición de una institución educativa con el fin de que sus estudiantes y profesores consigan el mayor desarrollo intelectual y personal posible, dentro los tiempos marcados.

Los documentos en los que nos hemos basado y han definido el progreso de la aplicación son: el estándar de calidad para moocs \cite{quality-reference-framework} que nos ha dado la base para hacer un desarrollo ordenado y razonado.Este marco define distintas dimensiones (fases, perspectivas y roles) que abarcan el proceso de la enseñanza a distancia desde que se que comienza el análisis para la creación de un curso hasta que el curso termina. El estudio de modelos de calidad de elearning \cite{modelos-calidad-elearning} es otro documento que define claramente la calidad en la docencia en línea y da ejemplos de modelos de elearning de calidad. En este documento pueden ver los resultados positivos en docencia en línea de varias instituciones académicas y los sistemas de educación de los que disponen para lograr esos resultados.

Este proyecto ha pasado por varias versiones anteriormente. Con las nuevas versiones de Moodle hemos encontrado problemas de compatibilidad por lo que en esta nueva versión hemos habilitado la aplicación para que pueda funcionar con las versiones 4.x de Moodle, además hemos solucionado problemas de portabilidad y aumentado funcionalidad.

En este documento, se explican todos los aspectos relacionados con el proyecto, tanto la parte práctica como la teórica. Mostraremos el proceso de desarrollo, los aspectos técnicos y las contingencias encontradas durante el mismo, además, se presentan conclusiones con los aspectos importantes a mencionar. 


